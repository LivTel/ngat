\documentclass[10pt,a4paper]{article}
\pagestyle{plain}
\textwidth 16cm
\textheight 21cm
\oddsidemargin -0.5cm
\topmargin 0cm

\title{ngat.spec package}
\author{C. J. Mottram}
\date{}
\begin{document}
\pagenumbering{arabic}
\thispagestyle{empty}
\maketitle
\begin{abstract}
This document describes classes in the ngat.spec package.
These classes wrap the libspec library, which controls the Manchester Eschelle/NuView spectrographs.
\end{abstract}

\centerline{\Large History}
\begin{center}
\begin{tabular}{|l|l|l|p{15em}|}
\hline
{\bf Version} & {\bf Author} & {\bf Date} & {\bf Notes} \\
\hline
0.1 &              C. J. Mottram & 19/10/00 & First draft \\
\hline
\end{tabular}
\end{center}

\newpage
\tableofcontents
\listoffigures
\listoftables
\newpage

\section{Documentation Overview}
\subsection{What is the ngat.spec package?}
These classes wrap the {\em libspec} library, using the JNI interface to communicate with the C routines.
The {\em libspec} library is a shared C library containing routines to communicate with either the
Manchester Eschelle Spectrograph (MES), or the NuView spectrograph.
Documentation on {\em libspec} is available at \cite{bib:libspec}.

The ngat.spec library consists of:
\begin{itemize}
\item The SpecLibrary class.
\item The SpecDoubleRange class.
\item The SpecIntRange class.
\item The SpecNativeException class.
\end{itemize}

\section{Class Overview}
API documentation for these classes is available at \cite{bib:ngatspecapi}.

\subsection{The SpecLibrary class}
This is the main class of the package. It contains all the methods that actually send commands to the CCD Controller
hardware. A static code section in the class loads the {\em libspec} shared library, which must be present in the
{\em LD\_LIBRARY\_PATH} for the class to be loaded.
The class has several main components:

\begin{itemize}
\item A set of public class constants, usually used as arguments to the methods. 
	These are mainly re-definitions of constants
	in the header files of {\em libspec}, and should be kept up to date with these definitions.
\item A set of private native methods. These have no implementation in Java, and when invoked call the
	relevant C routine using the JNI interface. They are usually named identically with the C functions they call,
	with underscores (`\_') in them. These calls of course go via the JNI code in {\em libspec}, which pack
	and unpack any Java data to C. The JNI code in {\em libspec} also causes these routine to throw an instance of
	SpecNativeException when the underlying {\em libspec} code returns an error value. This means {\em libspec}
	error routines do not have to be invoked from the Java layer, an Exception is returned in response to errors
	like `normal' Java methods.
\item A set of public methods. This is the main callable interface of this package. There is a public method
	for each of the {\em libspec} routines wrapped by the library, which just invoke the underlying
	private native method. Any SpecNativeException thrown in the native method is thrown up to the application
	layer to deal with.
\end{itemize}

\subsection{The SpecDoubleRange class}
This class allows various methods in SpecLibrary to return a double range, i.e. two double values min and max. 
It has a constructor such that it can be constructed from {\em libspec}'s JNI layer (from C).

\subsection{The SpecIntRange class}
This class allows various methods in SpecLibrary to return an integer range, i.e. two integer values min and max. 
It has a constructor such that it can be constructed from {\em libspec}'s JNI layer (from C).

\subsection{The SpecNativeException class}
This is an exception, which is thrown from all the public methods which call a native method that can fail.
This allows SpecLibrary to wrap the error routines in {\em libspec}. The error routines should not
be called from the Java layer, the native methods will automatically throw an instance of this class
when an error occurs in {\em libspec}.

\section{Using the package}
\subsection{Threads}
Because these routines are ultimately communicating directly with hardware, it is important to
consider what happens when several of these routines are running concurrently in different threads. These
routines communicate with two pieces of hardware:
\begin{itemize}
\item The Apogee AP7 Camera on the parallel port.
\item An A/D IO card via a device driver.
\end{itemize}
Neither of these pieces of hardware support more than one thing talking to them at once, therefore it is 
important not to call two SpecLibrary routines at once from different threads. Note there are some exceptions to this,
especially routines that abort procedures underway. The full list of concurrent capable routines is documented
in the {\em libspec} documentation \cite{bib:libspec}.

\subsection{Instantiation of SpecLibrary}
The SpecLibrary class uses all class (static) methods to access the Spectrograph. This is because the 
underlying {\em libspec} is designed to communicate with one instance of a spectrograph only (i.e. it holds all it's
spectrograph data internally, rather than via some application data structure). Therefore it does not make
sense to have two instances of a class that connects to the library, so all the methods are class methods.

\section{Conventions/Abbreviations}
\begin{itemize}
\item API - Application Programming Interface. The set of routines the library provides.
\item JNI - Java Native Interface. A Java mechanism for communicating with shared C libraries.
\item JVM - Java Virtual Machine. A program to actually run the Java program.
\item MES - Manchester Eschelle Spectrograph. A spectrograph.
\item RCS - Revision Control System. A set of operating system commands for controlling developer access to
source files. Type {\em man rcsintro} for details.
\end{itemize}

\begin{thebibliography}{99}
\addcontentsline{toc}{section}{Bibliography}
\bibitem{bib:libspec}{\bf Spectrograph Control System Library}
Liverpool John Moores University \newline{\em http://ltobs5.livjm.ac.uk/\verb'~'dev/spec/latex/libspec.ps}
\bibitem{bib:ngatspecapi}{\bf Ngat.Spec Generated API Documentation}
Liverpool John Moores University \newline{\em http://ltobs5.livjm.ac.uk/\verb'~'dev/ngat/javadocs/ngat/spec/index.html}
\end{thebibliography}

\end{document}
