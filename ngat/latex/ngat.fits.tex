\documentclass[10pt,a4paper]{article}
\pagestyle{plain}
\textwidth 16cm
\textheight 21cm
\oddsidemargin -0.5cm
\topmargin 0cm

\title{ngat.fits package}
\author{C. J. Mottram}
\date{}
\begin{document}
\pagenumbering{arabic}
\thispagestyle{empty}
\maketitle
\begin{abstract}
This document describes classes in the ngat.fits package.
These classes allow writing to FITS format files, via the CFITSIO library.
\end{abstract}

\centerline{\Large History}
\begin{center}
\begin{tabular}{|l|l|l|p{15em}|}
\hline
{\bf Version} & {\bf Author} & {\bf Date} & {\bf Notes} \\
\hline
0.1 &              C. J. Mottram & 26/05/00 & First draft \\
0.2 &              C. J. Mottram & 01/06/00 & Second draft \\
\hline
\end{tabular}
\end{center}

\newpage
\tableofcontents
\listoffigures
\listoftables
\newpage

\section{Documentation Overview}
\subsection{What is the ngat.fits package?}
These classes wrap the {\em CFITSIO} library, using the JNI interface to communicate with the C routines.
The classes allow programs to write keywords and their associated values to a FITS header. 
The classes allow programs to impose ordering of keywords within the header, and allow default values to
be specified in a configuration file for keyword values, comments, unit strings and ordering. The classes
only deal with writing the headers to disk, image data is written in the Ccs by using the {\em libccd}
\cite{bib:libccd} library.

The ngat.fits library consists of:
\begin{itemize}
\item The FitsHeader class.
\item The FitsHeaderCardImage class.
\item The FitsHeaderCardImageOrderNumberComparator class.
\item The FitsHeaderDefaults class.
\item The FitsHeaderException class.
\end{itemize}
 
\section{Directory Structure}
The {\em ngat.fits} package is part of the {\em ngat} package, and hence resides in the {\em fits} sub-directory
of that package. This \LaTeX documentation resides in the {\em latex} subdirectory of the {\em ngat} package.
The JNI code resides in the {\em c} and {\em include} sub-directories and the {\em fits} directory.
All this is shown in Figure \ref{fig:ngat.fits.dirtree}.

\setlength{\unitlength}{1in}
\begin{figure}[!h]
	\begin{center}
		\begin{picture}(5.0,7.0)(0.0,0.0)
			\put(0,0){\special{psfile=ngat.fits.dirtree.eps   hscale=65 vscale=65}}
		\end{picture}
	\end{center}
	\caption{\em ngat.fits directory tree.}
	\label{fig:ngat.fits.dirtree}
\end{figure}

\section{Class Overview}
API documentation for these classes is available at \cite{bib:ngatfitsapi}.

\subsection{The FitsHeader class}
This is class is the main part of the package. It performs two primary functions. It maintains a list
of FITS card images (keyword/value combinations), that can be cleared, added to and sorted. It
provides an interface, via the Java Native Interface, to C routines in the CFITSIO library that
allow the headers to be written to a FITS file on disk.

\subsection{The FitsHeaderCardImage class}
This class holds the data relevant to one card image in the header, and has get and set methods for each
element. The {\em FitsHeader} class maintains a list of instances of this object. The following data
is held :
\begin{itemize}
\item {\bf Keyword}.
\item {\bf Value}. This is held as an object reference, as different keywords have different types of values.
	The {\em FitsHeader} class only allows a limited sub-set of sub-classes of Object to be valid value classes.
	Currently the following classes are supported: String, Integer, Float, Double, Boolean, Date.
\item {\bf Comment}. FITS headers allow a comment field. This string is the comment, or it can be null.
\item {\bf Units}. CFITSIO allows the comment to contain a units field. This string is that field, or it can
	be null.
\item {\bf Order Number}. This is a number that is used to sort a list of {\em FitsHeaderCardImage} instances,
	so that they appear in the required order in the FITS header.
\end{itemize}

Note this class redefines it's {\em equals} method. Two instances of this class are equal if their
keywords match. This allows us to find and update instances of this class in {\em FitsHeader} easily,
using {\em java.util.List}'s {\em indexOf} method to find instances in the list by keyword.

\subsection{The FitsHeaderCardImageOrderNumberComparator class}
An instance of this class is used internally in the {\em FitsHeader} class. This is a comparator that
compares FitsHeaderCardImage instances by the orderNumber. It is used to
order the cardImageList by orderNumbers of the card images in it, so that
FITS header keywords are written in the right order.

\subsection{The FitsHeaderDefaults class}
\label{sec:fitsheaderdefaultsclass}
This class holds default data relating to the list of FITS header card images that makes up one FITS file. 
The defaults are read from a Java property file. See Section \ref{sec:configfile} for the format of the 
property file. 

The Java property file contains information on the following FITS card image attributes: 
\begin{itemize}
\item Types of values for keywords. 
\item Default values for keywords with constant values (over the execution of a program). 
\item Default comments for keywords. 
\item Default units for keywords. 
\item Default orderNumber for keywords. This determines where the keywords are placed in the FITS header, 
	which is important for mandatory keywords. 
\end{itemize}

After creating an instance of this class, one of the {\em load} methods is called to load the defaults from a 
Java property file. 
There are methods to get the defaults. Note the method to get the default value is quite complicated, as it
uses Java language reflection to construct an instance of a Java class from the type and value fields.

\subsection{The FitsHeaderException class}
This class is a sub-class of Exception. It is thrown from various methods in {\em FitsHeader} and
{\em FitsHeaderDefaults} when errors occur.

\section{JNI shared library}
The JNI shared library is generated from the source code in the {\em c} sub-directory. It is basically a 
set of procedure stubs that convert native methods in the {\em FitsHeader} class into CFITSIO function calls.
The procedures are declared in {\em include/ngat\_fits\_FitsHeader.h} and defined in {\em c/ngat\_fits\_FitsHeader.c}.
The {\em include/ngat\_fits\_FitsHeader.h} header file and it's associated declarations are automatically 
generated by {\em javah} from the {\em FitsHeader} native method declarations.
The API of the shared library is documented at \cite{bib:ngatfitsjniapi}.

In most cases, a native method in {\em FitsHeader} is implemented as follows:
\begin{itemize}
\item Convert Java parameter types into equivalent C types.
\item Call CFITSIO routine(s) to perform the requested operation.
\item Package up and returned values into Java types.
\item If an error occurs, throw an exception.
\end{itemize}

The writing of float/double values in FITS header keywords deserves note. Most keyword insert/updates are done
with {\em fits\_update\_key}. However, keywords of type float/double that are updated with this function
always use exponent mantissa form which is not what is required. Therefore this is only called if
the double value is larger than {\em FITS\_HEADER\_MIN\_EXPONENT\_NUMBER}, otherwise 
{\em fits\_update\_key\_fix[flt|dbl]} is called which puts the value into the header with {\em FITS\_HEADER\_DECIMALS}
decimal places.

\section{Configuration File}
\label{sec:configfile}
The configuration file is used by the {\em FitsHeaderDefaults} class (see Section \ref{sec:fitsheaderdefaultsclass}) 
to get defaults for fields that make up card images in the FITS header (fields in {\em FitsHeaderCardImage}).
This class recognizes properties with the key prefixes described in Table \ref{tab:defaultkeyprefix}.
These keys are all suffixed with the keyword string of the card image the defaults are queried for.

\begin{table}
\begin{center}
\begin{tabular}{|l|l|p{12em}|l|p{17em}|}
\hline
{\bf Default} 			& {\bf Key prefix} 	& {\bf Notes} \\ \hline
Types of values for keywords	& ngat.fits.value.type. & This must be fully qualified valid Java class name. \\ \hline
Default values 			& ngat.fits.value. 	& The value of keywords of this type must be able to be 
					passed into a String constructor for the value type defined above. \\ \hline
Default comments 		& ngat.fits.comment. 	& The value can be left blank, 
							if no comment is required. \\ \hline
Default units 			& ngat.fits.units. 	& The value can be left blank, 
							if no unit field is required. \\ \hline
Default orderNumber 		& ngat.fits.order\_number. & This should be an integer value. \\ \hline
\end{tabular}
\end{center}
\caption{\em Key prefixes in the defaults property file.}
\label{tab:defaultkeyprefix} 
\end{table}

Here is an example configuration file, showing the mandatory keywords in a FITS file:

\begin{verbatim}
# SIMPLE keyword
ngat.fits.value.type.SIMPLE 	=java.lang.Boolean
ngat.fits.value.SIMPLE 		=true
ngat.fits.comment.SIMPLE	=A valid FITS file
ngat.fits.units.SIMPLE		=
ngat.fits.order_number.SIMPLE	=0

# BITPIX keyword
ngat.fits.value.type.BITPIX	=java.lang.Integer
ngat.fits.value.BITPIX 		=16
ngat.fits.comment.BITPIX	=Bits per pixel
ngat.fits.units.BITPIX		=bits
ngat.fits.order_number.BITPIX	=1

# NAXIS keyword
ngat.fits.value.type.NAXIS	=java.lang.Integer
ngat.fits.value.NAXIS 		=2
ngat.fits.comment.NAXIS		=Number of axes
ngat.fits.units.NAXIS		=
ngat.fits.order_number.NAXIS	=2

# NAXIS1 keyword
ngat.fits.value.type.NAXIS1 	=java.lang.Integer
ngat.fits.value.NAXIS1 		=
ngat.fits.comment.NAXIS1	=
ngat.fits.units.NAXIS1		=pixels
ngat.fits.order_number.NAXIS1	=3

# NAXIS2 keyword
ngat.fits.value.type.NAXIS2 	=java.lang.Integer
ngat.fits.value.NAXIS2		=
ngat.fits.comment.NAXIS2	=
ngat.fits.units.NAXIS2		=pixels
ngat.fits.order_number.NAXIS2	=4
\end{verbatim}

\section{Conventions/Abbreviations}
\begin{itemize}
\item API - Application Programming Interface. The set of routines the library provides.
\item FITS - Flexible Image Transport System. The file format used to store images taken by instruments
	on the telescope.
\item JNI - Java Native Interface. A Java mechanism for communicating with shared C libraries.
\item JVM - Java Virtual Machine. A program to actually run the Java program.
\item RCS - Revision Control System. A set of operating system commands for controlling developer access to
source files. Type {\em man rcsintro} for details.
\end{itemize}

\begin{thebibliography}{99}
\addcontentsline{toc}{section}{Bibliography}

\bibitem{bib:ngatfitsapi}{\bf ngat.fits Generated API Documentation}
Liverpool John Moores University \newline{\em http://ltccd1.livjm.ac.uk/\verb'~'dev/ngat/javadocs/ngat/fits/index.html}

\bibitem{bib:ngatfitsjniapi}{\bf ngat.fits JNI C API Documentation}
Liverpool John Moores University \newline{\em http://ltccd1.livjm.ac.uk/\verb'~'dev/ngat/cdocs/ngat/fits/index.html}

\bibitem{bib:libccd}{\bf CCD Control System Library}
Liverpool John Moores University \newline{\em http://ltccd1.livjm.ac.uk/\verb'~'dev/ccd/latex/libccd.ps}
\end{thebibliography}

\end{document}
