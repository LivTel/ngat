\documentclass[10pt,a4paper]{article}
\pagestyle{plain}
\textwidth 16cm
\textheight 21cm
\oddsidemargin -0.5cm
\topmargin 0cm

\title{ngat.math package}
\author{C. J. Mottram}
\date{}
\begin{document}
\pagenumbering{arabic}
\thispagestyle{empty}
\maketitle
\begin{abstract}
This document describes classes in the ngat.math package.
These classes perform various mathematical operations required for various sub-systems of the telescope.
\end{abstract}

\centerline{\Large History}
\begin{center}
\begin{tabular}{|l|l|l|p{15em}|}
\hline
{\bf Version} & {\bf Author} & {\bf Date} & {\bf Notes} \\
\hline
0.1 &              C. J. Mottram & 30/08/00 & First draft \\
\hline
\end{tabular}
\end{center}

\newpage
\tableofcontents
\listoffigures
\listoftables
\newpage

\section{Documentation Overview}
\subsection{What is the ngat.math package?}
This package is a collection of mathematical classes for performing various math operations that various
sub-systems for the telescope needs. Currently, this consists of classes for fitting quadratic curves
to a series of data points.

The ngat.math library consists of:
\begin{itemize}
\item The ChiSquaredFit class.
\item The ChiSquaredFitDataValuer interface.
\item The ChiSquaredFitModeller interface.
\item The ChiSquaredFitUpdateListener interface.
\item The QuadraticFit class.
\item The QuadraticFitPoint class.
\end{itemize}

There are also some test programs that supply test data to the QuadraticFit class to test the accuracy of it's
results. These are the QuadraticFitTest and QuadraticFitTest2 classes.

\section{Class Overview}
API documentation for these classes is available at \cite{bib:ngatmathapi}.
The API documentation for the test classes is at \cite{bib:ngatmathtestapi}.

\subsection{The ChiSquaredFit class}
Class to do a chi squared fit, which returns chi squared for a model verses an actual data set. 
Chi squared is a measure of how well the model fits the data. 

\subsection{The ChiSquaredFitDataValuer interface}
Interface describing a model to call when calculating a Chi Squared Fit. 
This interface allows the Chi squared fit to get an actual data value for a given index in the list of data objects. 
Implementing classes must implement the following method:
\begin{verbatim}
double getValue(java.util.List dataList,int index);
\end{verbatim}
This method returns the value of the data at the given index in the list of data. 

\subsection{The ChiSquaredFitModeller interface}
Interface describing a model to call when calculating a Chi Squared Fit. 
The model can have various parameters which are set to a value for one run through the chi squared fit. 
This interface extends the ChiSquaredFitDataValuer interface, with the getValue method now returning
the model's calculated value at the index. It adds the following method
\begin{verbatim}
void setParameter(java.lang.String name, double value);
\end{verbatim}
This method is called from ChiSquaredFit to set one of the model's parameters (specified by name) to
the specified value.

\subsection{The ChiSquaredFitUpdateListener interface}
Interface used to call an update function after each new chi squared iteration, 
before new parameter limits are computed.
The following method is called by the ChiSquaredFit at various points during the fitting process:
\begin{verbatim}
void chiSquaredUpdate(int type, ChiSquaredFit csf);
\end{verbatim}
The {\bf csf} parameter is the ChiSquaredFit object reference that is currently doing the processing
(allowing the listener to query it), whilst the {\bf type} number determines whilst the update listener was called.
The {\bf type} number can take the following values:
\begin{itemize}
\item {\bf UPDATE\_TYPE\_BEST\_CHI\_SQUARED} Called whenever the current parameter set has produced the 
best (smallest) chi-squared value when testing the model against the data.
\item {\bf UPDATE\_TYPE\_CHI\_SQUARED} Called after every parameter iteration with the current
values/chi-squared values in place.
\item {\bf UPDATE\_TYPE\_FINISHED} Called when chi-squared fit has finished the loop, either because a good
match has occurred or the maximum number of parameter constraints has been achieved.
\item {\bf UPDATE\_TYPE\_PARAMETER\_LOOP} Called after a parameter loop has been run. The best 
parameter values found in this loop are used to constrain the parameter ranges for the next loop.
\end{itemize}

\subsection{The QuadraticFit class}
\label{sec:quadraticfitclass}
This class does a quadratic fit for a series of (x,y) data points. 
It works out values (a,b,c) for a quadratic fit such that:
\begin{math}
y = a \times x^2 + b \times x + c 
\end{math}
This is done using an instance of the ChiSquaredFit class mentioned above. There are methods to
determine a,b and c by two different algorithms:
\begin{itemize}
\item A three degree of freedom single chi squared fit, that iterates through a three-dimensional parameter space.
\item Three single degree of freedom chi squared fits. The first calculates the {\bf a} parameter by calculating
        the second derivative of the data points (by calculating the rate of change of gradient between 
        three points) and using the model {\bf 2a}. 
	The second chi squared fit calculates the {\bf b} parameter using the first derivative of the data points 
	(their gradient) and using the model {\bf 2ax +b}. 
	The third chi squared fit calculates the {\bf b} parameter using the actual y positions of the data points
	and the full model $y = a \times x^2 + b \times x + c$.
\end{itemize}
There are a series of inner classes that implement the various ChiSquaredFitModeller's and ChiSquaredFitDataValuer's
needed to do the above. These are:
\begin{itemize}
\item {\bf QuadraticChiSquaredFitActual} implements ChiSquaredFitDataValuer. 
	Private inner class to get the actual value of y from the data. The getValue method returns y.
\item {\bf QuadraticChiSquaredFitModel} implements ChiSquaredFitModeller. 
	Private inner class to model $y = a \times x^2 + b \times x + c$. 
	i.e. a,b and c are the parameters allowed by setParameter. 
\item {\bf QuadraticChiSquaredFitGradientActual} implements ChiSquaredFitDataValuer. 
	Private inner class to get the the first derivative of 
	$y = a \times x^2 + b \times x + c$, from the actual data. 
	The getValue method returns the rate of change of gradient. Because the calculated gradient occurs
	{\bf between} two points, it actually returns the average gradient of the data point before this
	point and this point, and this point to the data point after it. This means you must be careful of the
	index passed to getValue.
\item {\bf QuadraticChiSquaredFitGradientModel} implements ChiSquaredFitModeller. 
	Private inner class to model the first derivative of 
	$y= a \times x^2 + b \times x + c$.
	i.e. $value = 2 \times a \times x + b$. A and B are parameters, 
	A can be set in the constructor so a 1 degree freedom of movement can be used once A has been determined. 
\item {\bf QuadraticChiSquaredFitGradientRateActual} implements ChiSquaredFitDataValuer. 
	Private inner class to get the the second derivative of 
	$y = a \times x^2 + b \times x + c$, from the actual data. 
	The getValue method returns the rate of change of gradient. This assumes the index passed in has a point either
	side of it, and returns the difference in gradient between the two sets of points. Again, care must be
	taken to supply index's greater than zero and {\bf two} less than the count of data points, so that
	the getValue method can use the previous and next index's safely.
\item {\bf QuadraticChiSquaredFitGradientRateModel} implements ChiSquaredFitModeller. 
	Private inner class to model the second derivative of 
	$y= a \times x^2 + b \times x + c$. i.e. value = 2a. A is the only acceptable parameter.
\end{itemize}

\subsection{The QuadraticFitPoint class}
This class holds an x/y position of a data value, that we are trying to obtain a fit for. The data list in
QuadraticFit is a list of instances of this class.

\subsection{The QuadraticFitTest class}
Simple test class for QuadraticFit. Puts command line arguments as y values in data list (i,args[i]). 
Then does a quadratic fit and prints the result out. 

\subsection{The QuadraticFitTest2 class}
Simple test class for QuadraticFit. Command line arguments are a,b,c,perturbation factor,number of points. 
Then does a quadratic fit and prints the result out. 

\section{Quadratic Fit Algorithm}
The purpose of a quadratic fit is to get a function, $y = f(x)$ such that $f(x)$ matches as closely as possible
some set of data points. We assume for a quadratic fit that the points lie roughly in a curve, i.e. the
rate of change of gradient of a line between the data points is effectively constant. If the data does not match
this criteria a different order polynomial should be used. For a quadratic fit the equation of the line
is $y = a \times x^2 + b \times x + c$, and we are searching for parameters a,b, and c that cause the line to
match the data as closely as possible.

The best way to search for a match is to use a chi-squared fit. This compares a set of data points
to a model, and calculates an error term for each data point which is summed. This chi-squared value can
be computed for a set of potential parameter values, and the set with the lowest chi-squared value is the
closest to the data. The set of parameter values can then be constrained around the best value to try
to find a better approximation.

There are two algorithms used to get a Quadratic Fit. 

The first uses one three degree of freedom
chi-squared fit, where the parameter space to search is one large three dimensional array of a,b and c parameters.

However, if we differentiate the equation of the line twice, we get the second differential to be:
\newline
\begin{math}
\frac{d^2y}{dx^2} = 2 \times a
\end{math}
\newline
This allows us to do one degree of freedom chi-squared fit of a, against a computed
value for the second differential from the data, which can be obtained with three data points as shown in
Figure \ref{fig:secdiffdata}.

\setlength{\unitlength}{1in}
\begin{figure}[!h]
	\begin{center}
		\begin{picture}(3.0,3.0)(0.0,0.0)
			\put(0,0){\special{psfile=ngat.math.secdiffdata.eps hscale=90 vscale=90}}
		\end{picture}
	\end{center}
	\caption{\em How the second differential is calculated from the data.}
	\label{fig:secdiffdata}
\end{figure}

After finding a, we can plug it back into the first differential of the equation of a quadratic to do
another one degree of freedom chi-squared fit on b, using:
\newline
\begin{math}
\frac{dy}{dx} = 2 \times a \times x + b
\end{math}
\newline
We can get the data values for the first differential, by finding the gradient between two points:
\newline
\begin{math}
\frac{dy}{dx} = \frac{y2 - y1}{x2 - x1}
\end{math}
\newline
The gradient found in this case tends to be one located between the two points.
It was found to be better to use three points, one before and one after the actual value, and
use the average gradient, as this better represents the gradient at the actual point in consideration.

\section{Pseudo-code}
The pseudo-code shown here is for the one three degree of freedom algorithm, as this is the most
complicated.

The QuadraticFit class should be constructed. This sets the parameter start ranges to their default values.
Various configuration methods can then be applied to configure the fit:
\begin{itemize}
\item The {\em setParameterStepCount} can be used to set the number of steps to take between the minimum and maximum
	values for each parameter. Note this only takes effect after the first parameter constraint.
\item The {\em setParameterStartValues} can be used to set the initial range and step size for each of the
	parameters a,b, and c. It is important to choose good initial ranges to get a good fit.
\item The {\em setUpdateListener} can be called to set a listener that is called when various events
	occur during the fitting process.
\end{itemize}

The data points to fit to should then be specified using the {\em addPoint} method (overloaded to give
several ways of specifying the data).

The {\em quadraticFit} method should then be called to do the fit.
It takes two parameters, {\em loopCount} and {\em targetChiSquared}. The loop count is used to
determine the maximum number of times to constrain the parameter ranges, and the target chi-squared
stops constraining the parameter ranges early, if the current fit's chi-squared value is less than the
target. The resulting values of a,b, and c can then
be returned using the {\em getA}, {\em getB} and {\em getC} methods. The {\em getChiSquared} method returns a measure
of how good the fit is (the nearer to zero the better).

\subsection{quadraticFit method}
This method actually does the fit. In the one three degree of freedom algorithm, it uses one
instance of ChiSquaredFit class to do the fit. This instance is constructed, and the data is set to
the list passed in. The start and end elements of the list to fit against are set to zero and the 
size of the list. The modeler is set to the {\em QuadraticChiSquaredFitModel} 
(see Section \ref{sec:quadraticfitclass}). The actual data return valuer is set to 
{\em QuadraticChiSquaredFitActual}. The parameter's a,b,c are added to the chi-squared fit, which is then run,
using the {\em chiSquaredFit} method.

\subsection{chiSquaredFit method}
This has two parameters, maxLoopCount and targetChiSquared.
\begin{verbatim}
initialise variables.
while(not finished)
        call parameterLoop(0)
        increment loop count
        finished = (loop count > maximum loop count) or ( bestChiSquared < targetChiSquared)
\end{verbatim}

\subsection{parameterLoop}
This method has a parameter, which is the parameter index. It is a recursive function, that
calls itself for each parameter until they have all been set, when it calls the {\em dataLoop} method.
\begin{verbatim}
get information on parameter at the parameter index.
for each parameter value from start value to end value step by the step size
        Calculate this parameter's value.
        Sets the model's value for this parameter
        if (this paramater Index is not the last in the list)
                call parameterLoop(parameter index + 1)
        else
                call dataLoop
\end{verbatim}

\subsection{dataLoop}
\begin{verbatim}
call getChiSquared
if returned chi-squared is less than best chi-squared
        best chi-squared = chi-squared
        for each parameter in the parameter list
                set the parameter's best value to it's current value.
\end{verbatim}

\subsection{getChiSquared}
Internal method to do one data sum loop and return a chi squared value.
Note dataModeller.setParameter should have been called for each parameter before this is called.
\begin{verbatim}
for each index between the data start index and the data end index
        get the model's value for this data/parameter set.
        get the actual data value for this data.
        increment the chi squared with the difference between the model and actual value.
\end{verbatim}

\section{Conclusion}
The ngat.math package allows us to do reasonable fits of quadratic curves to data points. The quality of fit
is dependent of the quality of initial parameter ranges supplied to the fit. The three one degree of freedom 
algorithm is a lot quicker than the one three degree of freedom algorithm. The three one degree of freedom 
algorithm produces better results for data with small deviations from a curve, but for large deviations it
can fail quite miserably, whilst the one three degree of freedom algorithm will still get a reasonable 
result. This is because the three one degree of freedom 
algorithm does not find a good {\bf a} parameter, because the second derivative data calculation does
not produce a very straight line. This then causes the {\bf b} and {\bf c} parameter fits to fail.

\section{Conventions/Abbreviations}
\begin{itemize}
\item API - Application Programming Interface. The set of routines the library provides.
\item RCS - Revision Control System. A set of operating system commands for controlling developer access to
source files. Type {\em man rcsintro} for details.
\end{itemize}

\begin{thebibliography}{99}
\addcontentsline{toc}{section}{Bibliography}
\bibitem{bib:ngatmathapi}{\bf Ngat.Math Generated API Documentation}
Liverpool John Moores University \newline{\em http://ltccd1.livjm.ac.uk/\verb'~'dev/ngat/javadocs/ngat/math/index.html}
\bibitem{bib:ngatmathtestapi}{\bf Ngat.Math Test Program API Documentation}
Liverpool John Moores University 
\newline{\em http://ltccd1.livjm.ac.uk/\verb'~'dev/ngat/javadocs/ngat/math/test/index.html}
\end{thebibliography}

\end{document}
