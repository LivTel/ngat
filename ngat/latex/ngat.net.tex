\documentclass[10pt,a4paper]{article}
\pagestyle{plain}
\textwidth 16cm
\textheight 21cm
\oddsidemargin -0.5cm
\topmargin 0cm

\title{Java Message System: ngat.net package}
\author{C. J. Mottram}
\date{}
\begin{document}
\pagenumbering{arabic}
\thispagestyle{empty}
\maketitle
\begin{abstract}
This document describes classes in the ngat.net package.
These classes deal with sending information over the network.
\end{abstract}

\centerline{\Large History}
\begin{center}
\begin{tabular}{|l|l|l|p{15em}|}
\hline
{\bf Version} & {\bf Author} & {\bf Date} & {\bf Notes} \\
\hline
0.0 &              C. J. Mottram & 4/5/99 & First draft \\
\hline
\end{tabular}
\end{center}

\newpage
\tableofcontents
\listoffigures
\listoftables
\newpage

\section{Documentation Overview}
\subsection{What is the ngat.net package?}
This package is a set of Java classes that send information over a network.
It consists of:
\begin{itemize}
\item The TCPServer class.
\item The TCPServerConnectionThread class.
\item The TCPClientConnectionThread class.
\end{itemize}

\section{Design Rationale}
There are several different mechanisms to communicate data over a network that we considered at the design stage.

\subsection{Simple Socket connection}
This is the simplest interface. It could be written in either C or Java and would probably be the most efficient 
communication methods with the minimum of overhead. However there would be a significant overhead in abstracting 
the information out of objects into memory to send across the network. 
Also, when the contents of the objects change then the code to abstract the information out of the objects would 
also have to change: two pieces of code (at least) would have to change when the contents of a message changes.
It would be easy for the two pieces of software not to be up to date leading to errors.

\subsection{Remote Method Invocation(RMI)}
RMI is the Java version of rpc's (remote procedure calls), where you effectively call a remote method on another 
machine. The problem with this method of communication is that it limits the information that can be returned from 
the remote method, and it does not allow a two way communication dialogue to be set up easily.

\subsection{Object Serialization}
This is another Java mechanism for sending Java objects over any stream interface (a socket connection being a 
stream interface). Transferring an object simply involves one line of code as does receiving an object. 
Serialization creates a tree of all objects referenced by the root object, simplifying the data transmission process. 
However, care has to be taken that the classes at both ends of the connection are the same version otherwise 
serialization can fail (this is probably a good thing). 

\subsection{Mechanism Selection}
For out first prototype we have selected Object Serialization as our mechanism for object transfer. 
It seems to be the simplest mechanism for our needs. The simple socket connection will involve development of
a lot of support code, which will be difficult to keep up to date. RMI is fine for simple communication, but will
soon become too complicated when two processes need interactive two way communication.

\section{Protocol}
The protcol used is documented in the Java Message System Design Document \cite{bib:jmsdd}.

Figure \ref{fig:ngatnetprotocol} shows how a transaction takes place within the system:
\setlength{\unitlength}{1in}
\begin{figure}[!h]
	\begin{center}
		\begin{picture}(5.0,5.5)(0.0,0.0)
			\put(0,0){\special{psfile=ngat.net.protocol.eps   hscale=70 vscale=75}}
		\end{picture}
	\end{center}
	\caption{\em ngat.net example transaction. }
	\label{fig:ngatnetprotocol} 
\end{figure}

\section{Class Overview}
These are the classes currently available that implement the protocol previously described. API documentation
for these classes is also available \cite{bib:ngatnettree}.

\subsection{The TCPServer class}
The class implements a TCP Server. The class extends the standard Java Thread class, so that the resulting Server
will sit on it's own thread. The thread creates a SocketServer which sits on a defined port awaiting connections. 
When a connection is accepted, a new thread of class TCPServerConnectionThread is spawned to deal with this connection.
This class should be sub-classed so that a process specific derivative of TCPServerConnectionThread is spawned
to deal with the specifics of passing messages to that process.

\subsection{The TCPServerConnectionThread class}
This class implements the generic message protocol used by the message passing system. It is instantiated as a thread,
and an object of this class is run for each connection accepted. The generic implementation involves receiving a 
command, sending an acknowledge message back, and then sending a done message to complete the transaction. Sub-classes
of this class should process the command passed in and perform the command, and also calculate the time to complete 
a command.

\subsection{The TCPClientConnectionThread class}
This class implements the client end of the generic message protocol. It is instantiated as a thread,
and an object of a sub-class of this class is run for each command to be sent. The generic implementation involves
opening a socket connection to the server, sending a command to the server, awaiting an acknowledgement, 
and then awaiting a done message to complete the transaction. This class is sub-classed to provide the process
specific implementations to implement internal responses to the acknowledgement and the done message.

\begin{thebibliography}{99}
\addcontentsline{toc}{section}{Bibliography}
\bibitem{bib:jmsdd}{\bf Liverpool Telescope Java Message System Design Document}
Liverpool John Moores University
\bibitem{bib:ngatnettree}{\bf Ngat.Net Class Tree}
Liverpool John Moores University \newline{\em http://ltobs5.livjm.ac.uk/devngatjavadocs/ngat/net/tree.html}
\end{thebibliography}

\end{document}
