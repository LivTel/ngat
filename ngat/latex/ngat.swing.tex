\documentclass[10pt,a4paper]{article}
\pagestyle{plain}
\textwidth 16cm
\textheight 21cm
\oddsidemargin -0.5cm
\topmargin 0cm

\title{Java Message System: ngat.swing package}
\author{C. J. Mottram}
\date{}
\begin{document}
\pagenumbering{arabic}
\thispagestyle{empty}
\maketitle
\begin{abstract}
This document describes classes in the ngat.swing package.
These classes are common utility classes used when writing GUIs using the javax.swing.* package.
\end{abstract}

\centerline{\Large History}
\begin{center}
\begin{tabular}{|l|l|l|p{15em}|}
\hline
{\bf Version} & {\bf Author} & {\bf Date} & {\bf Notes} \\
\hline
0.0 &              C. J. Mottram & 29/11/99 & First draft \\
\hline
\end{tabular}
\end{center}

\newpage
\tableofcontents
\listoffigures
\listoftables
\newpage

\section{Documentation Overview}
\subsection{What is the ngat.swing package?}
This package is a set of Java utility classes for performing common operations
when writing GUIs using the javax.swing.* package.
It consists of:
\begin{itemize}
\item The GUIDialogUnmanager class.
\item The GUILabelSetter class.
\item The GUITextAppender class.
\item The MinimumSizeFrame class.
\item The TitledSmallerBorder class.
\end{itemize}


\section{Swing Features}
The javax.swing package currently has features which require these utility classes to get around.

\subsection{Threading}
All method calls that update graphical elements in the swing package must be called from the swing thread in Java.
This is because the Swing package does not synchronize it's method calls within the package because of
the resulting performance degredation.

This causes problems when we wish to update our GUIs outside the Swing thread. This is a command problem in our GUIs
as inter-process communication is done using threads to allow us to service more than one command at once. The ngat.net
package, which implemnts the Java Message System (JMS) we use for inter-process communication, is thread based. We
want to write receipt of acknowledments and command completeion messages to the parts of our GUI that
reflects the current state.

Luckily, Swing allows for the fact that users of the toolkit may want to perform these operations. It provides a 
SwingUtilities class, which has an {\em invokeLater} method. This takes a {\em Runnable} object as a parameter,
and runs this objects {\em run} maethod from within the Swing thread, so it is thread safe. This package contains
several {\em Runnable} classes for use with this mechanism.

\section{Class Overview}
API documentation for these classes is available at \cite{bib:ngatswingtree}.

\subsection{The GUIDialogUnmanager class}
The GUIDialogUnmanager is Runnable. It is used as an argument to SwingUtilities.invokeLater.
It unmanages a JDialog. This is needed when a dialog is unmanaged from a menu item, as otherwise
the menu-bar the item is on complains about events from unmanaged components.

\subsection{The GUILabelSetter class}
The GUILabelSetter is Runnable. It is used as an argument to SwingUtilities.invokeLater.
It sets a label's text. This should be used when a label needs to be updated out-side of the Swing
thread.

\subsection{The GUITextAppender class}
The GUITextAppender is Runnable. It is used as an argument to SwingUtilities.invokeLater.
It appends some passed in text to the JTextArea. This should be used when some text needs to
be appended to a JTextArea out-side of the Swing thread.

\subsection{The MinimumSizeFrame class}
This class extends JFrame, and tries to set a minimum size for the frame.
It does this by monitoring Component Window re-sizeing events, and checking the current size	
against the minimum size and re-sizing if appropriate. This is not a very good mechanism
for doing this, as the frame re-sizes smaller before this class has a chance to make it bigger again.

\subsection{The TitledSmallerBorder class}
This class extends TitledBorder, but tries to reduce the ridiculously big gap between
the title text and the inner contents. It does this by over-riding {\em getBorderInsets}.
This may not be needed once the font settings of our delevopment system are sorted out.

\begin{thebibliography}{99}
\addcontentsline{toc}{section}{Bibliography}
\bibitem{bib:ngatswingtree}{\bf Ngat.Swing Class Tree}
Liverpool John Moores University \newline{\em http://ltccd1.livjm.ac.uk/\verb'~'dev/ngat/javadocs/ngat/swing/index.html}
\end{thebibliography}

\end{document}
